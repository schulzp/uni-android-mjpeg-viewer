\section{Interviews}

\subsection{Interview with FD (40, m) 23.3.2016 19:30}
\label{appendix:interview_f_d}

FD started by describing a wholistic approach to climbing and factors influencing it. There are key
qualities like maximum strength (\enquote{Maximalkraft}), cardio (\enquote{Ausdauer}), mental strength (\enquote{Psyche}), explosive strength (\enquote{Schnellkraft}), flexibility (\enquote{Beweglichkeit}), and motivation (\enquote{Motivation}). An ideal climber might have all of these qualities or at least those needed for reaching a certain long term goal. Such a goal here might be climbing multi pitch big walls or winning a competition. Succeeding in reaching the goal, is the main source of motivation. 

FD's training process is based on that model. Once the goal is known, the important qualities can be derived and evaluated for deficits. It is a trainers job to figure out the deficits of a trainee through observation and interpretation, and come up with methods to reduce them, which the actual training is about. This sequence of evaluation and training is repeated.

According to FD, both, trainee and trainer should be aware of factors having an influence on the
training success. He names the following factors: 

\begin{itemize}
\item diet (\enquote{Ernährung})
\item regeneration (\enquote{Schlafen/Pause})
\item variability in methods (\enquote{Variabilität der Trainingsmethoden})
\item environmental factors/context (\enquote{Trainingsumfeld}).
\end{itemize}

It is up to the trainer to observe the trainees and read the condition they are in, which again requires empathy and sensitivity to the trainees needs. If training does not yield the expected results, the cause might be a factor which is not directly related to the training process, but nonetheless may have an influence on it.

Depending on how serious the trainee is about a goal, and depending on the trainee's age, the trainer might has to control the training process more actively. Independent from the level of autonomy, --- for those trainees who can mostly train themselves --- FD highly recommends writing a training log to monitor the progress. Further he introduces the idea of an app which could work as a training log and helps analyzing the logged data by visualizing it. Since smartphones are widely available the hurdle of logging would be lowered, even more since it saves the logger from transferring the logging data from one medium to another. Going even further, the app could suggest training methods based on the knowledge about the trainee's deficits.

Another method already applied by FD is video analysis. He records a trainee while climbing and they evaluate the recording together afterwards. FD tries to figure out what might be a possible cause for an unsuccessful try and articulates his assumptions. It is up to the trainees to compare this observation with their own impression of how they felt. This method of self reflection can result in improved self-awareness and in improved movement anticipation. However, this method better works with more experienced climbers who are open to suggestions, and not as much for younger trainees, about the age of 16, who need more active guidance.

\subsection{Interview with IL (44, f) 5.5.2016 22:30}
\label{appendix:interview_i_m}

IL is a former  successful competitive lead climber who works as a professional trainer for sport climbing today. She coaches private customers as well as students and youth groups at a competitive level.

IL got introduced to alpine sports in Russia by her parents at a young age. Climbing --- especially sport climbing --- was not widely known back then, which is why she had to teach herself most of the necessary skills when she endeavored to become a mountain guide. In 2000, she migrated to West Germany where she started training for sport climbing, since 2002 coached by a professional trainer. She became part of the climbing squad of North Rhine-Westphalia and started working as a trainer herself in 2005.

She credits all her achievements to her discipline. Getting better requires a goal and a strong will to achieve it, especially overcoming one's weaker self and go to the limit instead. This is the motto IL lives by and expects from her trainees.

According to IL, climbing is a very complex sport because it involves a variety of complex movements combined with an essential mental component. Hence there can't be a training that fits all since success or failure depends on many individual factors.

When IL starts working with a new customer or group, the first thing she does is getting to know them. She lets them climb to see what might be potential deficits and interviews them about their training goals or fears they want to overcome. IL does not rely on training plans as she thinks they are too complex for beginners and inflexible, mainly because there are so many training aspects which can be picked and combined based on the mental/physical condition of the day. Instead she gives advice what to work on and leaves it to her trainee to put them into action.

IL describes several techniques of detecting and communicating deficits. One of these techniques is talking the trainee through a route he or she is currently struggling with, step by step. If the trainee succeeds with the live instructions, this means his or her struggle is not an issue of strength but might rather be an issue of tactics. Another approach is discussing what happened right after a trainee dropped off the wall, sometimes supported by video analysis. So far, IL used her smartphone to record videos of someone climbing and analyzed the playback either immediately afterwards, together with the subject, or later to take notes.

Her main concern regarding training is a lack of information for beginners. There are many books and other resources on the internet, yet, of them target experts or are simply too hard to understand and follow for newcomers. Even worse, some of the exercises suggested pose the risk of injury if not performed correctly or too early in the training process. IL sees it as her responsibility as an expert to provide any trainee with a suitable set of training methods and oversee their proper execution. The least she hopes for is an app or some resource which can be accessed on mobile devices giving detailed instructions using video and audio which are targeted at multiple expertise levels.


\subsection{Interview with. FL (25, m) 6.5.2016 17:40}
\label{appendix:interview_f_l}

FL is still an active competitive climber who started coaching others a few years ago.

FL gained experience with video analysis as trainee as well as trainer. Today he is rather critical about this method and only uses it occasionally for detail analysis. Video analysis requires a lot of storage space and a lot of  time is needed to analyze the playback.

A precise perception of the own body is the key to good climbing. From FL's experience video analysis is not very effective in conveying this awareness. However, letting trainees reflect and describe their pose on the wall and what when wrong, immediately after they dropped off the wall, helps them to build up a vocabulary and improves the awareness for their body. This method is called imagery.
